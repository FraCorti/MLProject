\section{Method}

\subsection{"Code"}

Briefly (short part) describe what you developed and how:
• The code (for type A implementation) or the used simulator(s) (for type B). 
◦ The used tools/libraries (if any)
◦ Software overview and the software design choices (if interesting)
◦ Implementation choices: a  summary of the choices (e.g. architecture/s, training algorithm/s, type of activation function/s, batch/on-line/mb, regularization schema, stop condition)
◦ The novelties (if any) but not the standard approaches (do not describe the algorithms/models we already described in the lectures). Use references for the source of information. 
• Preprocessing procedure (if any) [details may be postponed to Section 3]
• Validation schema (model selection and evaluation schema) for the Experimental part: report data splitting  TR/VL/TS (% data for each set and/or the K values of the k-fold CV) [details may be postponed to Section 3]
• Type of preliminary trials pursued (often summarized by text) [details may be postponed to Section 3]



Each figure/table should be referenced as in the following, see Fig. 1. 
Do not use figure/table without a number. Do not write “see the next figure” (which one?).
Tables and plots have always a caption. All of the Figures and Tables should be cited in order, including those in the Appendix. (which should be cited as, for example, Fig. A.1, and Table A.1).